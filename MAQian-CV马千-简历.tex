% !TEX TS-program = xelatex
% !TEX encoding = UTF-8 Unicode
% !Mode:: "TeX:UTF-8"

\documentclass{resume}
\usepackage{zh_CN-Adobefonts_external} % Simplified Chinese Support using external fonts (./fonts/zh_CN-Adobe/)
% \usepackage{NotoSansSC_external}
% \usepackage{NotoSerifCJKsc_external}
% \usepackage{zh_CN-Adobefonts_internal} % Simplified Chinese Support using system fonts
\usepackage{linespacing_fix} % disable extra space before next section
\usepackage{hyperref}
\usepackage{cite}
\usepackage{xcolor}

\begin{document}
\pagenumbering{gobble} % suppress displaying page number

\name{马千}

\basicInfo{
  \email{qma233-c@my.cityu.edu.hk} \textperiodcentered\ 
  \phone{(+86) 18081180675} \textperiodcentered\ 
  % \linkedin[billryan8]{https://www.linkedin.com/in/billryan8}
  }
 
\section{\faGraduationCap\  {教育背景}}
\datedsubsection{{\href{https://portland-my.sharepoint.com/:f:/g/personal/qma233-c_my_cityu_edu_hk/EvgUYxJ87_hImoWnjgUrumMBBdONb_MJCfrogSKQKWOjBQ?e=ISmUIC}{\color{blue}{\textbf{电子科技大学}}}}, 成都, 四川}{2018 -- 2022}
\textit{学士}\ 软件工程-系统与技术
\datedsubsection{\textbf{香港城市大学}, 香港}{2022 -- 至今}
\textit{在读硕士研究生}\ MscMIT, 预计 2023 年 8 月课程结束(论文项目要求持续到Summer Term结束)


\section{\faUsers\ 实习经历}
\datedsubsection{\href{https://portland-my.sharepoint.com/:f:/g/personal/qma233-c_my_cityu_edu_hk/EvgUYxJ87_hImoWnjgUrumMBBdONb_MJCfrogSKQKWOjBQ?e=ISmUIC}{\color{blue}{\textbf{SAP Labs China}}} 西安}{2021年4月 -- 2022年5月}
\role{实习}{经理: {JingtaoLi} {jing-tao.li@sap.com} }
Develop Intern in the Dev-Infra Team of \href{https://www.sap.com/products/technology-platform/data-intelligence.html}{\color{blue}{SAP Data Intelligence}}
\begin{itemize}
  \item 云基础设施服务自动化平台 Cloud Infrastructure service automation and Continuous
  \item 持续集成交付自动化平台 Continuous Integration(CI)/Continuous Delivery (CD) automation.
  \item 实习期间我是平台中两个微服务(Micro-service)的具体负责人
  \item 有多次作为主讲人在内部的全球分享会上演示介绍由我主持的我们组的新项目或者新增功能的经历,观众主要是全球同二级架构下的 L2-Org SAP T\&I HANA Database and analytics的同事
  \item 因为产品是SaaS所以我们组除了自己的平台还会和具体开发组件功能的组一起合作来优化架构在云原生层面的实现
\end{itemize}



\section{\faUsers\ 研究经历}
\datedsubsection{\textbf{电子科技大学}}{2019年12月 -- 2022年6月}
在苏生老师的实验室接触了有关城市计算Urban Computing的研究课题。
作为大二的本科生主要负责了数据处理方面的工作,有点像通过行驶记录生成grid-based data的预处理。
大概的难点如下:
\begin{itemize}
  \item Python全局解释锁(GIL)导致效率比较低只能单核
  \item 很多操作是可以并行的,比如计算GeoHash对应的经纬度,起始时间对应的time-interval的index
  \item 所以我实现了个类似MapReduce的方案,来利用全部CPU核心并行处理
\end{itemize}

\datedsubsection{\textbf{\href{https://aml-cityu.github.io/team/}{\color{blue}{AML Cityu}}}}{2022年10月 -- 至今}
以前的实习经历让我对参与开发的Data Intelligence软件背后的理论基础和相关研究产生了兴趣,这也是我渴望继续深造的原因之一。
我选择了论文项目来满足毕业要求,现在我在城大SDSC的\href{https://aml-cityu.github.io/team/}{AML}实验室,接受\href{https://www.cityu.edu.hk/stfprofile/xyzhao.htm}{\color{blue}{赵翔宇}}老师的指导,和我的Phd师兄们一起参与智慧城市相关课题的研究。

% Reference Test
%\datedsubsection{\textbf{Paper Title\cite{zaharia2012resilient}}}{May. 2015}
%An xxx optimized for xxx\cite{verma2015large}
%\begin{itemize}
%  \item main contribution
%\end{itemize}

\section{\faInfo\ 英语水平}
% increase linespacing [parsep=0.5ex]
\begin{itemize}[parsep=0.5ex]
  \item IELTS:7.0(L:7.0 R:8.5 W:6.0 S:5.5)
  \item CET-6:551
\end{itemize}
\section{\faCogs\ Coding 技能}
% increase linespacing [parsep=0.5ex]
\begin{itemize}[parsep=0.5ex]
  \item 编程语言: Python > Java == JavaScript == Golang > C
  \item 平台: Linux
  \item 开发工具: Git, Docker , Kubernetes , Singularity(HPC 构建镜像来定制环境)
  \item CI/CD: Jenkins
  \item DL框架: PyTorch
  \item 论文写作: \LaTeX
\end{itemize}



\section{\faHeartO\ 获奖情况}
\datedline{\textit{校标兵奖学金}}{2018-2019学年}
\datedline{\textit{校标兵奖学金}}{2020-2021学年}
\datedline{\textit{校标兵奖学金}}{2021-2022学年}
\datedline{\textit{西南赛区一等奖}{第四届``长风杯''全国大学生大数据分析与挖掘竞赛}}{2020年12月}
\datedline{\textit{全国优胜奖}{第四届``长风杯''全国大学生大数据分析与挖掘竞赛}}{2020年12月}



\section{\faInfo\ 关于招聘要求的对应}
% increase linespacing [parsep=0.5ex]
因为我的目标是24年的phd申请,所以我就对照您的博士招生要求再突出下。

关于前五条:
% \datedsubsection{关于前五条}
\begin{itemize}[parsep=0.5ex]
  \item 我的本科是软件工程背景,和我们学校CS的培养方案并无太大差别,具体课程可以查看\href{https://portland-my.sharepoint.com/:f:/g/personal/qma233-c_my_cityu_edu_hk/EvgUYxJ87_hImoWnjgUrumMBBdONb_MJCfrogSKQKWOjBQ?e=ISmUIC}{\color{blue}{我的本科成绩单}}。
  \item 我对时空数据挖掘和城市计算的课题感兴趣,也很希望能做出有意义的科研成果所以选了论文项目毕业的方案。
  \item 数学,编程能力(包括python,Linux使用)这些我都自认为还不错,也愿意学习探索新知识。
  \item 十分认同科研需要有开源、开放、合作的精神,不管是之前实习还是现在进行研究的时候,我们也都会向一些不错的开源项目学习并做些贡献。
  \item 英语基础的话我的雅思写作成绩6.0,口语成绩5.5,满足了科大对小分的要求但可能没满足您的`雅思写作/口语在6分以上'的要求(如果/表示AND的话)但我不认为这个口语成绩能反映了我的真实水平。因为实习期间我经常会和北美、欧洲(主要是德法)、巴西的同事们开会,也经常做presentation,大家对我的评价都还挺好的,印象也蛮深刻的,还有不少同事会后找我单独交流的(可能也有因为我名字里的Qian比较少见也比较难发音的原因,所以国外同事印象会更深一点吧,后来我Teams英文名用的Van)。所以我有自信把我实习Manager的邮箱也写在CV上。
\end{itemize}
关于第六条的多个要求:
\begin{itemize}
  \item 暂无自己的一作长文发表,但选了论文项目接受了赵老师组的指导正在准备CIKM23的投稿,希望能中
  \item 实习时我的部门属于SAP的T\&I(Technology\&Innovation),总体是负责研发的部门,我不是直接负责开发AI/DM部分的,但还是有些基础的理解,因为我也参与了他们的pipeline搭建。
  \item 丰富的全栈工程师经验,我认为这应该是我相比大部分同龄人比较突出的方面,虽然实习的一年我主要是负责后端模块,但因为有在学校综合课设对前端的基础了解,所以对应的前端修改还是没问题的,比如给新加的功能做个单独的页面。测试也是软件工程的一部分,所以我也会,之前实习开发的CI/CD平台的主要任务之一也是实现自动化测试。
\end{itemize}
再总结下我的优势的话,
\begin{itemize}
  \item 我已经有接受过相关的基础训练,所以不需要您再投资更多的时间来教导基础,比如Latex写作、PyTorch使用、环境配置等琐碎的基础(当然我自己总结的文档也乐意分享给团队)。
  \item 我认为自己是个不错的team-player,在大企业敏捷开发的架构下能有在保证自己工作完成的情况下对其他人的工作能有基本的认知来为他人提供desired input的意识,能对您创建团队有所帮助。
  \item 因为您是要从零开始搭建自己的团队,所以我认为之前培养出来的工程师精神和流程管理的意识能节省您的精力,比如环境配置等琐碎工作按照我之前实习学到的流程我就可以写成Blog放到wiki里供所有人查阅(之前城大开始用HPC,我们实验室的怎么自己定制化开发环境镜像的文档就是我写的。)
\end{itemize}
我非常期望能加入您的团队来做出更多有影响力的工作。
%% Reference
%\newpage
%\bibliographystyle{IEEETran}
%\bibliography{mycite}
\section{\faUsers\ 项目经历}
综合课程设计是我们学院相较于计算机学院的一个特色课程,主要目标是锻炼学生的写作能力和项目能力,要求实现完整的软件工程的过程(V-Model这种完整的开发测试流程)

对您来说应该不如前面的部分重要,但可供参考。
\datedsubsection{\textbf{综合课程设计1}}{2019年2月 -- 2019年6月}
\role{Python,OpenCV}{个人美颜相机}
\begin{onehalfspacing}
\begin{itemize}
  \item 可以利用软件工程的基本知识进行需求分析、系统设计和实现
  \item 协同开发基于OpenCV的人脸识别模块
  \item 协同集成美颜算法
\end{itemize}
\end{onehalfspacing}

\datedsubsection{\textbf{综合课程设计2/3}}{2019年8月 -- 2019年12月 \& 2020年2月 -- 2020年6月}
\role{JavaScript, Python}{基于爬虫的网络动漫素材管理系统}
\begin{onehalfspacing}
综合课程设计3时主要是进行了一些网络网面的优化提升了实际使用体验
\begin{itemize}
  \item 构建容器化的统一开发环境
  \item 利用系统分析和设计的技能,进行需求分析、系统设计和实现
  \item 设计实现关系型数据库来存储各种素材信息,将数据库部署在云服务器上
\end{itemize}
\end{onehalfspacing}
\end{document}

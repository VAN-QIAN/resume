% !TEX TS-program = xelatex
% !TEX encoding = UTF-8 Unicode
% !Mode:: "TeX:UTF-8"
\documentclass{resume}
\usepackage{zh_CN-Adobefonts_external} % Simplified Chinese Support using external fonts (./fonts/zh_CN-Adobe/)
% \usepackage{NotoSansSC_external}
% \usepackage{NotoSerifCJKsc_external}
% \usepackage{zh_CN-Adobefonts_internal} % Simplified Chinese Support using system fonts
\usepackage{linespacing_fix} % disable extra space before next section
\usepackage{hyperref}
\usepackage{cite}
\usepackage{xcolor}
\usepackage{etoolbox}
\patchcmd{\thebibliography}{\section*{\refname}}{}{}{}

\begin{document}
\pagenumbering{gobble} % suppress displaying page number

\name{Qian Ma}

\basicInfo{
  \email{maq5@rpi.edu} \textperiodcentered\ 
  \phone{(+86) 18081180675} \textperiodcentered\ }

\section{\faGraduationCap\ {Education}}
\datedsubsection{{\textbf{Rensselaer Polytechnic Institute}}}{Current}
\textit{Ph.D. Student} in Computer Science
\datedsubsection{{\href{https://portland-my.sharepoint.com/:b:/g/personal/qma233-c_my_cityu_edu_hk/EaddzNGLGsRMoknymTwm_a8BE8Hy64B1wz7O_CJD4iYaKQ?e=XzIWeg}{\color{blue}{\textbf{City University of Hong Kong}}}}}{2022 -- 2023} %
\textit{Master's Degree} in MscMIT with Distinction
\datedsubsection{\href{https://portland-my.sharepoint.com/:f:/g/personal/qma233-c_my_cityu_edu_hk/EvgUYxJ87_hImoWnjgUrumMBBdONb_MJCfrogSKQKWOjBQ?e=ISmUIC}{\color{blue}{\textbf{University of Electronic Science and Technology of China}}}}{2018 -- 2022}
\textit{Bachelor's Degree} in Software Engineering - Systems and Technology


\section{\faUsers\ Research Experience}

\datedsubsection{\textbf{\href{(https://aml-cityu.github.io/team/)}{\color{blue}{AML Cityu}}}}{October 2022 - June 2023}
My previous internship experiences have sparked my interest in the theoretical foundations and related research behind Data Intelligence software development. This is one of the reasons why I am eager to continue my academic pursuits. Currently, I am working in the \href{[https://aml-cityu.github.io/team/ ↗](https://aml-cityu.github.io/team/)}{AML} laboratory at City University of Hong Kong (CityU SDSC), under the guidance of Professor \href{[https://www.cityu.edu.hk/stfprofile/xyzhao.htm ↗](https://www.cityu.edu.hk/stfprofile/xyzhao.htm)}{\color{blue}{Xiangyu Zhao}}. I am collaborating with my Ph.D. seniors on research projects related to smart cities.

\datedsubsection{\textbf{Rethinking Sensors Modeling: Hierarchical Information Enhanced Traffic Forecasting\cite{qma23}}}{August 2023}
\href{https://github.com/VAN-QIAN/CIKM23-HIEST}{\color{blue}{Project Link}}\\

I chose the EE6680D paper project, with guidance from Professor \href{(https://www.cityu.edu.hk/stfprofile/haoliali.htm)}{\color{blue}{Haoliang Li}} and Professor \href{(https://www.cityu.edu.hk/stfprofile/xyzhao.htm)}{\color{blue}{Xiangyu Zhao}}.

The paper project carries the same credit as three graduate courses. I successfully had the paper accepted by CIKM23 with an A+ for the dissertation project.

\begin{itemize}
\item Motivated the rethinking of spatial dependencies in spatio-temporal forecasting
\item Made core methodological contributions by constructing two new hierarchical perspectives and utilizing cross-level information enhancement to improve task performance at the original level
\item Independently designed and implemented all experiments
\item Finished the academic paper writing with the guidance from my advisors. 
\item Received positive feedback from rigorous peer reviewers and obtained acceptance
\end{itemize}

\datedsubsection{\textbf{University of Electronic Science and Technology of China}}{December 2019 -- June 2022}
In Professor Su Sheng's laboratory, I was involved in research projects related to Urban Computing.
As a sophomore undergraduate student, my main responsibility was data processing, which involved pre-processing grid-based data from driving records.
The following are some of the challenges I encountered:
\begin{itemize}
\item The Global Interpreter Lock (GIL) in Python resulted in lower efficiency as it only allowed single-core processing.
\item Many operations could be parallelized, such as computing the latitude and longitude corresponding to GeoHash and the time-interval index corresponding to the start time.
\item Therefore, I implemented a MapReduce-like solution to leverage all CPU cores for parallel processing.
\end{itemize}

\section{\faUsers\ Internship Experience}
\datedsubsection{\href{(https://portland-my.sharepoint.com/:f:/g/personal/qma233-c_my_cityu_edu_hk/EvgUYxJ87_hImoWnjgUrumMBBdONb_MJCfrogSKQKWOjBQ?e=ISmUIC)}{\color{blue}{\textbf{SAP Labs China}}} Xi'an}{April 2021 -- May 2022}
\role{Internship}{Manager: {Jingtao Li} {jing-tao.li@sap.com}}
Develop Intern in the Dev-Infra Team of \href{(https://www.sap.com/products/technology-platform/data-intelligence.html)}{\color{blue}{SAP Data Intelligence}}
\begin{itemize}
\item Cloud infrastructure service automation platform.
\item Continuous integration (CI) / continuous delivery (CD) automation platform.
\item During the internship, I was responsible for two micro-services in the platform.
\item I have presented and introduced our team's new projects or added features multiple times as the speaker at internal global sharing sessions. The audience mainly consisted of colleagues from L2-Org SAP T\&I HANA Database and Analytics, which is a global team under the second-level architecture.
\item As the product is SaaS, our team collaborated with other teams responsible for specific development components to optimize the implementation of cloud-native architecture.
\end{itemize}

\section{\faCogs\ Skills}
\begin{itemize}[parsep=0.5ex]
  \item Programming Languages: Python > Java == JavaScript == Golang > C
  \item Platform: Linux
  \item Development Tools: Git, Docker , Kubernetes , Singularity(to build customized enviornment for HPC)
  \item CI/CD: Jenkins
  \item DL Framework: PyTorch
  \item Paper Writting: \LaTeX
\end{itemize}

\section{\faInfo\ English Proficiency}
% increase linespacing [parsep=0.5ex]
\begin{itemize}[parsep=0.5ex]
  \item IELTS:7.0(L:7.0 R:8.5 W:6.0 S:5.5)
  \item CET-6:551
\end{itemize}

\section{\faHeartO\ Honors and Awards}
\datedline{\textit{Outstanding Student Scholarship}}{2018-2019 Academic Year}
\datedline{\textit{Outstanding Student Scholarship}}{2020-2021 Academic Year}
\datedline{\textit{Outstanding Student Scholarship}}{2021-2022 Academic Year}

\section{\faInfo\ Miscellaneous}
\begin{itemize}[parsep=0.5ex]
  \item GitHub: https://github.com/VAN-QIAN
  \item Languages: English - Fluent, Mandarin - Native speaker
\end{itemize}

\section{\faInfo\ Publications}
\bibliographystyle{IEEEtran}
\bibliography{mycite}
\end{document}
